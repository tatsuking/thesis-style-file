\documentclass[thesis]{thesis}
\usepackage{graphicx}
\usepackage[T1]{fontenc}
\usepackage{lmodern}
\usepackage{textcomp}
\usepackage{latexsym}
%\usepackage[fleqn]{amsmath}
%\usepackage{amssymb}

\FiscalYear{平成31年度} % 年度を入力

\newcommand{\AmSLaTeX}{%
 $\mathcal A$\lower.4ex\hbox{$\!\mathcal M\!$}$\mathcal S$-\LaTeX}
\def\BibTeX{{\rmfamily B\kern-.05em{\scshape i\kern-.025em b}\kern-.08em
 T\kern-.1667em\lower.7ex\hbox{E}\kern-.125em X}}

\jtitle{和文タイトル}
%\jsubtitle{技術研究報告原稿のための解説とテンプレート}
\etitle{English Title }
%\esubtitle{Guide to the Technical Report and Template}
\authorlist{%
 \authorentry[○○ ○○]{53xx}{電制 花子}% 
}
%
\begin{document}
\begin{jabstract}
ここには概要を書く。ここには概要を書く。ここには概要を書く。ここには概要を書く。ここには概要を書く。ここには概要を書く。ここには概要を書く。
ここには概要を書く。ここには概要を書く。ここには概要を書く。ここには概要を書く。ここには概要を書く。ここには概要を書く。ここには概要を書く。
ここには概要を書く。ここには概要を書く。ここには概要を書く。ここには概要を書く。ここには概要を書く。ここには概要を書く。ここには概要を書く。
ここには概要を書く。ここには概要を書く。ここには概要を書く。ここには概要を書く。ここには概要を書く。ここには概要を書く。ここには概要を書く。
\end{jabstract}
\begin{jkeyword}
キーワード1,キーワード2
\end{jkeyword}
\maketitle

\section{はじめに}



ここでは,本クラスファイルの使用にかかわる点のみを説明します.

レイアウトに関係するパラメータの変更などは行わないでください.
また,文字や段落の位置調節を行うための \verb/\vspace/,
\verb/\smallskip/,\verb/\medskip/,
\verb/\hspace/ などのコマンドの使用は必要最少限にとどめ,
\texttt{list} 環境のパラメータを変更することも避けてください.

\section{クラスファイルの説明}
\label{sec:cls}

\subsection{テンプレートと記述方法}
\label{sec:technicalreport}

以下のテンプレートに従って記述してください.
原稿執筆に際しては,本クラスファイルとともに配布される
テンプレート(\texttt{template.tex})を利用できます.

\begin{verbatim}
\documentclass[technicalreport]{ieicej}
\jtitle{卒業論文の書き方}
\etitle{How to Write a thesis }
\authorlist{%
 \authorentry[○○ ○○]{53xx}{電制 花子}% 
}
\begin{document}
\begin{jabstract}
概要
\end{jabstract}
\begin{jkeyword}
キーワード(5個程度)
\end{jkeyword}
\maketitle
本文
\end{document}
\end{verbatim}

\begin{itemize}
\item
「卒業論文」を書く場合はドキュメントクラスのオプションとして\texttt{thesis} を指定します.
「中間発表要旨」を書く場合はドキュメントクラスのオプションとして\texttt{midterm} を指定します.

\item
\verb/\jtitle/ には和文題目を指定します.
任意の場所で改行したいときは,\verb/\\/ で改行できます.


\item
発表者名および指導教員名は,以下のように記述します.
\begin{verbatim}
\authorlist{%
 \authorentry[指導教員]{学籍番号}{発表者名}
}
\end{verbatim}

\item
\verb/\label/ を記述する場合は,
必ず \verb/\caption/ の直後に置きます.
上におくと \verb/\ref/ で正しい番号を参照できません.
\end{itemize}

\subsubsection{図の取り込み}

図の取り込みに関しては,「技術研究報告」では,
「発表者が作成した原稿をそのままオフセット印刷します」ので,
図はどのような形式のものでも構いません.
ここではPDF形式の図を読み込む場合の説明を簡単にします.

例えば,パッケージとして
\begin{verbatim}
\usepackage[dvipdfmx]{graphicx}
\end{verbatim}
を指定し,
\begin{verbatim}
\begin{figure}[tb]
\begin{center}
 \includegraphics{file.pdf}
\end{center}
\caption{}
\ecaption{}
\end{figure}
\end{verbatim}
のような使い方をします.
\begin{thebibliography}{99}
%\bibitem{ohno}
%大野義夫編,\TeX\ 入門,
%共立出版,東京,1989. 

%\bibitem{Seroul}
%R. Seroul and S. Levy, A Beginner's Book of \TeX, 
%Springer-Verlag, New York, 1989. 

%\bibitem{nodera1}
%野寺隆志,楽々\LaTeX{},
%共立出版,東京,1990. 

\bibitem{Okumura3}
奥村晴彦,[改訂版]\LaTeXe\ 美文書作成入門,
技術評論社,東京,2000. 

\bibitem{FMi2}
% M. Goossens, S. Rahts, and  F. Mittelbach,  
% The \LaTeX\ Graphics Companion (Addison-Wesley, 1997).
マイケル・グーセンス,セバスチャン・ラッツ,フランク・ミッテルバッハ,
\LaTeX\ グラフィックスコンパニオン,アスキー出版局,東京,2000. 

\bibitem{FGo1}
% M. Goossens, and S. Rahts, 
% The \LaTeX\ Web Companion, Addison-Wesley,  1999.
マイケル・グーセンス,セバスチャン・ラッツ,
\LaTeX\ Web コンパニオン---\TeX\ とHTML/XML の統合,
アスキー出版局,東京,2001. 

\bibitem{PEn}
ページ・エンタープライゼス\<(株)\<,
\LaTeXe\ マクロ \& クラスプログラミング基礎解説,
技術評論社,東京,2002. 

\bibitem{Fujita5}
藤田眞作,\LaTeXe\ コマンドブック,
ソフトバンク パブリッシング,東京,2003. 

\bibitem{Yoshinaga}
吉永徹美,
\LaTeXe\ マクロ \& クラスプログラミング実践解説,
技術評論社,東京,2003. 

\bibitem{texwiki}
https://oku.edu.mie-u.ac.jp/\~{}okumura/texwiki/
\end{thebibliography}


\end{document}
